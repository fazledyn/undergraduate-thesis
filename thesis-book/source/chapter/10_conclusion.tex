\chapter{Conclusion}\label{conclusion}

We can conclude our research through the results found in chapter \ref{results}. The study conducted throughout this research dives into the interaction and other analytical data of social media along with a state-of-the-art Natural Language Processing-based approach. The findings show us that the food reviews shared on social media are taken in positive sentiment by most of the viewers whereas some outliers denote a negative sentiment towards the reviews.


\section{Future Work}
Since our study relies heavily on more and more data points, we intended to expand our dataset to get clearer insights. In the future, we hope to expand our work into different food review-related groups where we might find more insights regarding this industry that is increasing day by day. Since the overall process shown throughout our study is data-dependent and lengthy, we plan to incorporate an automated system through our developed plugin. We believe both users and the research community will benefit from the data collected through the plugin.

The interaction-based sentiment analysis introduced in paper \cite{paper_freeman} and \cite{paper_pratama} can be applied to other research areas working with social media. The special reactions classified into multiple categories along with a radar chart-based metric might help us look inside the data and provide rich insights. Being a cost and resource-efficient approach, this method needs more and more attention to it which can lead to an initial indication or hypothesis carried out at the initial phase of research.

We also wish to publish the dataset collected throughout this research. We believe it will not only add value to the existing datasets related to sentiment analysis but also will help expand the research of the Bangla language.
