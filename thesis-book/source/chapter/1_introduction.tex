\chapter{Introduction}\label{intro}

With the technological advancement in the 21st century, more and more people are getting internet access. As of January 2023, around 5.16 billion people in the world have internet access. In Bangladesh, around 38.9\% of the population have internet access. \cite{datareportal2023} Now that internet access has become easier than before, more and more people are joining social media. According to a report from \cite[DataReportal]{datareportal2023}, in 2023, around 43.25 million Bangladeshis are Facebook users. Facebook outperforms other social media sites like Twitter, and Instagram by a higher margin.

Nowadays, Facebook is not just a platform to connect with friends and family. Facebook itself has become a global village, for example- it's a place to get the latest news. Due to the newsfeed ranking algorithms, you do not have to manually search for any news by yourself. The news will come to you if you have liked any of the news profiles. In addition to that, Facebook has become an online marketplace where people from different walks of life are buying and selling items, making their livelihood out of this social media site. Furthermore, the rise of mass users on Facebook has given advertisements a new edge in the digital edge. Now anyone can post an advertisement on Facebook and pay the minimum to reach people of any demographic. Day by day, these new additions to the platform are increasing user engagement.

Facebook as a medium of entertainment is just another feature that is taking the world by storm. Every day tons of content are being uploaded to the platform, mostly in video format. Stories, reels, and videos are the main type of entertainment content that has a high engagement rate. Nowadays, content creators are not just limited to YouTube with their content. They are constantly sharing their content on Facebook and generating huge amounts of revenue from Facebook. On the other hand, people from all walks of life are getting easy access to content from their phones. Facebook has become their digital living room. Different types of content such as- Travel vlogs, cooking videos, food review videos, DIY (Do It Yourself) videos, real-life hacks, etc.




\section{Food Review Scenario in Bangladesh}
When it comes to creating content on social media, Bangladesh is no different. In recent years, there has been a rise in Bangladeshi social media influencers. Restaurants, food delivery services, catering businesses, etc. have contributed to the food review ecosystem in Bangladesh. Businesses are frequently engaging with influencers to help them generate traffic. On the other hand, customers also give attention to the influencers, while deciding for themselves. They trust the recommendation of the influencers and vloggers when it comes to deciding on a restaurant, a food item, or a food delivery service.


\subsection{Bangladeshi Top Food Vloggers and influencers}
The popularity of Bangladeshi food vloggers and influencers is increasing day by day. Mostly, influencers and vloggers rely on social media sites like Facebook and Instagram mainly. However, they have a notable presence on their YouTube channels also. Below, we are going to discuss some of the most popular food vloggers and influencers in this country.

\subsubsection{Iftekhar Rafsan}
Iftekhar Rafsan, who is mostly known as \textbf{Rafsan The Chotobhai}, is the most popular and well-known food influencer in Bangladesh. He started his food vlogging-related journey in 2017 on YouTube. Currently, he has more than 3.2 million followers on Facebook. Apart from Facebook, he also shares his food review-related content on his YouTube channel. As of 2023, he has around 1.48 million subscribers and 204 videos on his YouTube channel. In 2020, he received the "Top Content Creator of The Year 2019" award from Bangladesh's state ICT Minister, Zunaid Ahmed Palak.\cite{hypescout}

\subsubsection{Fahrin Zannat}
Fahrin Zannat, also known as \textbf{Khudalagse} is a food vlogger from Dhaka, Bangladesh. She started her journey in July 2019, through her YouTube channel. It is one of the fastest-growing food vlogging-related YouTube channels. Her unique way of presenting and showcasing skills sets her apart from other influencers. As of 2023, she has over 1.5 million followers on Facebook and 609 thousand subscribers on her YouTube channel. Her husband Salman Sadi plays a vital role in her content creation journey. Both of them are among the top influential couples in the content creation industry of Bangladesh.

\subsubsection{Rasif Shafiq and Ridhima Khan}
Rasif Shafiq and Ridhima Khan, mostly known as the \textbf{Petuk Couple} are among the most famous YouTubers of Bangladesh. Their food vlogging journey started in 2018, mostly through their traveling journeys. As of 2023, they have around 1.8 million followers on Facebook and 772 thousand subscribers on their YouTube channel. Even though they started their vlogging journey through traveling-related videos, soon they became one of the top food vloggers in this country. Both of them as a duo, bring joy to millions through their content and authentic reviews.

\subsubsection{Nusrat Islam Nikkita}
Nusrat Islam Nikkita, also known as \textbf{Zoltan BD} is a food vlogger from Dhaka. She is not just a food vlogger, but also a food enthusiast herself. Her enthusiasm about the other influencers and food vlogging inspired her to embark on this journey. Currently, she has around 688 thousand followers on Facebook and 66 thousand subscribers on her YouTube channel. Her unique way of storytelling sets her apart from other food vloggers.

\subsubsection{Shahriar Rabbir and Kazi Tahsin}
Shahriar Rabbir and Kazi Tahsin are the names of two friends who started their early days of food vlogging as \textbf{Metroman} back in 2017. Both of them are known for their humor and storytelling abilities in the videos. As a duo, they have been creating quality video content regularly. At this moment, they have more than 300 thousand followers on Facebook. Their YouTube channel is home to more than 250 thousand subscribers. They also have a presence on the social media site TikTok where they have more than 3000 active followers.



\section{Thesis Objective}
This thesis aims to study the overall food review ecosystem of social media in Bangladesh. Since no such work has been done before, we aim to conduct the study thoroughly. Even though our thesis is based on the sentimental analysis of the food review in social media, in this paper we have presented more insights regarding the overall situation that will help anyone conducting a study on this domain in the future.

Our thesis performs a sentimental analysis of the food review content shared on social media by the top food vloggers and influencers. In our study, we have applied two basic approaches. The first approach is based on the interactions and reactions of the audience on social media. In the second approach, we have used Natural Language Processing technique to analyze the sentiments of the food reviews through the comment sections of the contents. Since the food review industry of Bangladesh is based on social media, we have taken two such approaches to get a whole idea about the overall situation.

With the hope of overcoming the limitations, we have emphasized the quality of our dataset as well as the collection process. The objectives of the thesis can be described below-

\begin{enumerate}
    \item {
        Collect interactions and reactions related to data of the popular food vloggers and influencers' content.
    }

    \item {
        Collect the comments of the posts related to the collected interactions and reactions.
    }

    \item {
        Clean and pre-process the dataset for further tasks.
    }
    
    \item {
        Apply algorithms and machine learning techniques discussed in the methodology of the thesis to analyze the sentiment of the food reviews.
    }
    
    \item {
        Find notable insights from the data collected throughout the thesis and share the experimental results.
    }
\end{enumerate}


\section{Thesis Organization}
The rest of the thesis is organized as follows:

\begin{itemize}
    \item {
        \textbf{Chapter 2} discusses the existing research and works related to the problem definition. It contains a detailed literature review and lists the limitations carried out by the existing research.
    }

    \item {
        \textbf{Chapter 3} explains the methodology of our thesis in detail.
    }

    \item {
        \textbf{Chapter 4} contains the entire process of collecting the data and steps to create the dataset for our study.
    }

    \item {
        \textbf{Chapter 5} gives an elaborated step taken by us to pre-process the dataset for our research.
    }

    \item {
        \textbf{Chapter 6} covers the features of the data that we selected for our training and finding purposes.
    }

    \item {
        \textbf{Chapter 7} describes methods and machine learning models used by us for finding the results.
    }

    \item {
        \textbf{Chapter 8} gives an overview of the different evaluation metrics chosen by us to measure our models' performance and the overall findings.
    }

    \item {
        \textbf{Chapter 9} presents the results obtained from our research in a detailed manner.
    }

    \item {
        \textbf{Chapter 10} concludes the thesis, along with our findings, and suggestions on what can do next on this topic.
    }
\end{itemize}

