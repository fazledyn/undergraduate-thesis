\chapter{Related Works}\label{related_works}

There have been several works in the social media domain regarding sentiment analysis. Since social media is mainly dependent on user interactions and communications, the scope of sentiment analysis and natural language processing have always been present.

\begin{itemize}
    \item {
        In "Sentiment Analysis on Bengali Facebook Comments To Predict Fan's Emotions Towards a Celebrity"\cite{paper_fans_emotions} paper, the authors have performed sentiment analysis on the comments posted by different fans on a celebrity's Facebook page. In this work, they have applied various machine learning algorithms to predict the sentiments of the comments. The algorithms used by the authors are- \textit{Random Forest Classifier, Support Vector Machine, Neural Network, Naive Bayes Classifier}. In this study, the authors categorized the emotions into classes such as: "Surprised", "Abusive", "Angry", "Religious", "Happy", "Excited", "Sad".
    }
    \item {
        The paper titled "Hateful Speech Detection in Public Facebook Pages for the Bengali Language"\cite{paper_hate_speech} describes an approach to hate speech detection using machine learning techniques in the Bengali language. In this paper, some basic classifiers such as \textit{Stochastic Logistic Gradient Classifier, Logistic Regression along with Principle Component Analysis} were used along with \textit{Gated Recurrent Unit, Long Short Term Memory, and Word2Vec} models. The paper shows high performance for neural network-based models with an F1 score of 0.69 and an accuracy of 70\%.
    }
\end{itemize}

In the food review/recommendation domain, there has been noteworthy progress in terms of sentiment analysis. For example, the works mentioned below give an overview of the approaches usually taken by the authors-

\begin{itemize}
    \item {
        The paper titled "Sentiment Analysis Techniques on Food Reviews Using Machine Learning" \cite{paper_zomato_amazon} employs machine learning techniques such as \textit{Naive Bayes Classifier, Support Vector Machine Classifier, Decision Tree, K-Nearest Neighbors}. The paper focuses on the food review collected from Zomato and Amazon Foods. Based on the reviews and ratings submitted by customers, sentiment analysis performed on them gives an idea about the food recommendation ecosystem. 
    }

    \item {
        The paper titled "Sentiment Analysis on Food Review using Machine Learning Approach" \cite{paper_yelp} uses the famous Yelp review dataset in their research. Sentiment analysis using machine learning algorithms for English text was used for the research.
    }

    \item {
        The paper titled "Sentiment Analysis of Bengali Texts on Online Restaurant Reviews Using Multinomial Naive Bayes" \cite{paper_yelp_bn} follows the same approach as the previous paper. However, the Yelp review dataset was translated into Bangla. The paper applies \textit{Multinomial Naive Bayes} algorithm to perform sentiment analysis on the translated Bangla reviews.
    }
\end{itemize}

Based on a thorough study and a literature review of the existing papers and research, we conclude that no work has been yet performed on the social media domain of food review in the Bangla language context. As a result, our proposed thesis justifies itself as an unexplored topic in the Bangla language domain.
